\documentclass[a4paper]{article}
\usepackage[english]{babel}
\usepackage[utf8]{inputenc}
\usepackage[margin=1.15in]{geometry}
\usepackage{amsmath}
\usepackage{setspace}
\setlength{\parskip}{\baselineskip}
\setlength{\parindent}{0pt}

\usepackage{enumitem}

\title{QF603 Group Mini-Project 2}

\author{Group F}

\date{\today}

\begin{document}
	\maketitle
	
	\begin{abstract}
		In this project, we performed and analysed two simple linear regressions. The first is a regression of the daily log returns of the DJIA Index (dependent variable) ("DJIA") on the S\&P500 Index (independent variable) ("SP500), between 29$\textsuperscript{th}$ January 1985 and 17$\textsuperscript{th}$ October 2018. The second, is a regression of the annual log returns for the same two indices (observations taken at the last day of the year).
	\end{abstract} 
	
	\newpage
	\setcounter{secnumdepth}{1}
	\section*{Task 3: Regression of Daily Log Returns}
	
	\subsection{Regression Coefficients}
	\underline{Results}
	\begin{itemize}[nosep]
		\item Alpha = $\hat{a} = -0.000024$, Beta = $\hat{b} = 0.987703$
		\item Standard Distribution of Residual = $\sigma_{u_t} = 0.002944$
	\end{itemize}

	The regression can be expressed as $r_{(DJIA, t)} = -0.000024 +  0.987703 r_{(S\&P500, t)}$.
	
	As Beta is close to 1, the sensitivity of the DJIA daily log returns to the SP500's is also approximately 1.
	Also, because Alpha is close to 0, there is no ``consistent'' (systematic) difference between DJIA daily log returns and the SP500's.
	
	\subsection{Hypothesis Testing on Coefficients}
	\underline{Results}
	\begin{itemize}[nosep]
		\item T-statistics for $\hat{a} = -0.740308$, and for $\hat{b} = 339.973655$
		\item Degree of Freedom = $8500 - 2 = 8498$
		\item Null Hypothesis ($H_0$) are that $a=b=0$.
		\item Alternate hypothesis ($H_1$) are that $a\ne b\ne 0$
		\item Critical value at 5\% significance level = $\pm1.9602$
	\end{itemize}
	
	For $\hat{a}$, the test statistic lies within the 95\% confidence interval (i.e. does not exceed the 5\% critical level). Therefore, we do not reject the null hypothesis. The intercept coefficient $\hat{a}$ is not significantly different from 0 at the 5\% significance level.
	
	In contrast, for $\hat{b}$, the test statistic lies well beyond the 95\% confidence interval (i.e. significantly exceeds the 5\% critical level). Therefore, we reject the null hypothesis and accept the alternate hypothesis. The slope coefficient $\hat{b}$ is significantly different from 0 at the 5\% significance level.
	
	\subsection{Goodness of Fit}
	\underline{Results}
	\begin{itemize}[nosep]
		\item R-Squared $(R^2) = 0.931512$
		\item Adjusted R-Squared $(\textnormal{adj-}R^2) = 0.931504$
	\end{itemize}

	The $R^2$ value reports the degree to which our independent variable (the SP500 daily log returns) explains the variation of the dependent variable (the DJIA daily log returns). Since R-square is 0.931512, it means that over 90\% of the variation in the dependent variable is explained by the independent variable.
	
	$\textnormal{Adj-}R^2$ also measures the goodness of model fitting, but takes into consideration the number of independent variables in the model. $R^2$ will only increase or  stay the same when we add more independent variables, even if they do not have any relationship with the dependent variable. $\textnormal{Adj-}R^2$ on the other hand, will "penalize" the model for having excessive dependent variables that do not significantly improve the model. 
	
	The $\textnormal{adj-}R^2$ is always lower than the $R^2$ value. And, in our case, because there is only 1 independent variable, the $\textnormal{adj-}R^2$ and the $R^2$ values are relatively equal.
	
	\subsection{Other Findings}
	\underline{Results}
	\begin{itemize}[nosep]
		\item Jarque-Bera test stats = 3187.500000
		\item Degrees of Freedom = lalala
		\item Null Hypothesis $H_0 = 0$
		\item Alternate Hypothesis $H_1 \ne 0$
		\item Critical Chi-Square Value at ??\% significance level = ????
	\end{itemize}
	
	The JB test's null hypothesis is JB = 0. If the null hypothesis is not rejected, it indicates that the
	distribution of the errors are normally distributed (under a certain level of significance).
	
	\section*{Task 4: Regression of Yearly Log Returns}
	\label{sec:num2}
	
	\subsection{Regression Coefficients}
	\underline{Results}
	
	\subsection{Regression Coefficients}
	\underline{Results}
	
	\subsection{Hypothesis Testing on Coefficients}
	\underline{Results}
	
	
	\subsection{Goodness of Fit}
	\underline{Results}
	
	\subsection{Other Findings}
	\underline{Results}
	
	to be continue...
	to be continue...
	
\end{document}
